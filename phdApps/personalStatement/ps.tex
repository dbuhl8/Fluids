% Source: http://tex.stackexchange.com/a/150903/23931
\documentclass{article}
\usepackage[letterpaper,margin=1in]{geometry}
\usepackage{xcolor}
\usepackage{fancyhdr}
\usepackage{tgschola} % or any other font package you like

\pagestyle{fancy}
\fancyhf{}
\fancyhead[C]{%
  \footnotesize\sffamily
  \yourname\quad
  web: \textcolor{blue}{\itshape\yourweb}\quad
  \youremail}

\newcommand{\soptitle}{Personal Statement}
\newcommand{\yourname}{Dante Buhl}
\newcommand{\youremail}{dbuhl8@gmail.com}
\newcommand{\yourweb}{https://dbuhl8.github.io/website/}

\newcommand{\statement}[1]{\par\medskip
  \underline{\textcolor{blue}{\textbf{#1:}}}\space
}

\usepackage[
  colorlinks,
  breaklinks,
  pdftitle={\yourname - \soptitle},
  pdfauthor={\yourname},
  unicode
]{hyperref}

\begin{document}

\begin{center}\LARGE\soptitle\\
\large of \yourname\ (Applied Math PhD applicant for Fall---2024)
\end{center}

\hrule
\vspace{1pt}
\hrule height 1pt

\bigskip

My name is Dante Buhl, and my choice in applying for a PhD is a decision I'm making due to a long line of work and personal history, none of which has been relatively easy. Some of the many factors originate from my upbringing and family culture, while others are indicental to my experiences in college. Ultimately, they have culminated in my decision to keep learning and develope my knowledge further. 

I grew up on the edge of the Greater Sacramento Area in a town called El Dorado Hills, where its always a little hotter and drier than you would like, and there isn't all that much to do. As a result, I was always sort of bored; school never quite needed my full attention to pass, and I was without many hobbies. My parents weren't very wealthy either, and divorced soon after we lost our house in the Recession. As a result, I grew up moving between a lot of different places since they had to move very often. My mother's side of the family is Peruvian, with some of them living in Lima and some in California and New York. My father's side of the family is white, mostly from California and Michigan, and their family origins dating back to various parts of europe so time ago. Not much of that discribes me very well however. 

Like my mother who's had to work 2 full-time jobs to provide for the family, I've always been a very driven person. Though a lot of my time and attention to were not always committed as useful or productive things. I've never had a problem with going on for hours working thouroughly on a specific thing. That is what lead me to math in my own way. Math is something that given a fixed period of time, I could obsess over. And that obsession grew. Every time a new subject was taught, given a subject wasn't terribly proof-heavy, I would consume it rather rapidly. Moreover, I never found it sufficient to just learn something, I had to also know why. Upon learning the quadratic formula in highschool algebra for example, I then had to understand why it worked. After trying to prove it myself, my teacher guided me through the proof after class per my inquiry. And once I found calculus, it seemed a limitless world had been opened to me and I decided I wanted to study Mathematics in college.

Before moving to Santa Cruz, I had been working as a supervisor at a chain of restauarants called Rubios Coastal Grill. This was during my freshman year of college  and also during the height of the pandemic (once restaurants were allowed to open up again that is). This was one of the harder times in my life as I suppose it was for everybody. My dad lost his job and I didn't really have much contact with the world besides work, videogames, and online lectures. So of course, I started working a lot as work was my only physical interface with people. During my first quarter at UCSC for example, I was working around 55 hours a week. I remember often learning integration by myself from homework prompts and youtube, after my late closing shifts. I never actually went to class for calculus 2, but I learned the subject very well regardless. As one would expect from an overworked college student, I was terrible at keeping track of important dates. So bad in fact, that I missed both the midterm and the final for Calculus 2. Because of this, I failed the course my first time around, even though I had 97$\%$ in the course besides the exams. These sort of awkward, avoidable difficulties permeated my freshman year experience. Some of those difficulties are visible in my performance, some of them are not. 

By the middle of my undergrad in Mathematics at UCSC, I found myself somewhat without purpose. I had been studying math so long, just because I found it pleasant to study math, that I sort of forgot what I wanted to do with my life. I also realized that knowing how to compute some integrals wasn't what made you the coolest mathematician ever. I wanted something to specialize in, and something to be the best at. I remember looking at courses offered that would satisfy my major requirements and feeling so bored; why was this upper-division course just called ``Algebra''? It was then I realized that Mathematics was a subject so tedious and convuluted that it had lost touch with the real world. 

That same day, I saw the course offering ``Introduction to Fluid Dynamics'' from the Applied Math department, and I became wonderfully curious.  Curious enough such that I realized I wanted to make a change. I took a Dynamical Systems class in Fall 2022 from the Applied Math department as it satisfied a requirement in my current major, and I fell in love with the concept. Proofs and specificity were still part of the subject, but the air of the subject was different. There was a freedom in that course which was never really present in any of the math classes I took. Moreover, the professor for that very course happened to be a part of the Fluids group. Very suddenly, my prior wonder become something tangible, it became conversations. Before that quarter ended, those conversations turned into an application, and I applied to the department's masters program and enrolled in the graduate level dynamical systems course. 

Now, a year later, I'm in the first year of my accelerated masters program completeing a masters thesis on Stratified Turbulence in Stellar Flows with Pascale Garaud, having already taken 2 graduate level Fluid Dynamics courses at UCSC, and spending a summer in Towson's REU program with Herve Nganguia studying numerical models of propulsion in fluids at micro scales using machine learning. It is sort of amazing how fast your life can change when you find something that really sparks your interest like that. And yet, I feel I'm still missing something. This year will end before I know it and I don't want to end my academic career knowing that there is still more for me to learn in this world. There are simply more classes I want to take, more professors that I want to meet, and potentially more subjects to find and become enamored with. For this reason, I want to continue my education with a PhD in Fluid Dynamics. There are several types of deparments that study this. Some are under Applied/Computational Mathematics programs, while others are found in various Fluid Mechanics or Mechanical Engineering programs. 

Going into the future, the sort of work I would want to do would ideally be a fusion of mathematics and computer science. This seems to be the core of a lot of advanced applied mathematics and finds beautiful, if not tedious, manifestations in high-performance computing, numerical methods, and machine learning problems. This art has helped realize some of the more beautiful mathematical pieces in our time such as the chaotic Lorenz System, and the period doubling bifurcation rate of the Mandlebrot Set. The use of modern computing has enhanced the field of mathematics and adjacent topics greatly, and I wish to be a part of that future. 




\end{document}
