% Source: http://tex.stackexchange.com/a/150903/23931
\documentclass{article}
\usepackage[letterpaper,margin=1in]{geometry}
\usepackage{xcolor}
\usepackage{fancyhdr}
\usepackage{tgschola} % or any other font package you like

\pagestyle{fancy}
\fancyhf{}
\fancyhead[C]{%
  \footnotesize\sffamily
  \yourname\quad
  web: \textcolor{blue}{\itshape\yourweb}\quad
  \youremail}

\newcommand{\soptitle}{Personal Statement}
\newcommand{\yourname}{Dante Buhl}
\newcommand{\youremail}{dbuhl8@gmail.com}
\newcommand{\yourweb}{https://dbuhl8.github.io/website/}

\newcommand{\statement}[1]{\par\medskip
  \underline{\textcolor{blue}{\textbf{#1:}}}\space
}

\usepackage[
  colorlinks,
  breaklinks,
  pdftitle={\yourname - \soptitle},
  pdfauthor={\yourname},
  unicode
]{hyperref}

\begin{document}

\begin{center}\LARGE\soptitle\\
\large of \yourname\ (Applied Math PhD applicant for Fall---2024)
\end{center}

\hrule
\vspace{1pt}
\hrule height 1pt

\bigskip

My choice in applying for a PhD is a decision I'm making due to a long line of work and personal history to myself. I grew up on the edge of the Greater Sacramento Area, where its always a little hotter and drier than you would like, and there isn't all that much to do. I grew up sort of bored, school never quite needing my full attention to pass, and without too many hobbies. I started playing guitar somewhere in highschool and stopped playing frequency within the last two years. My parents were divorced pretty early in my childhood, and I grew up in a lot of different households since they had to move often. My mother's side of the family is Peruvian, with some of them living in Lima and some in California and New York. My father's side of the family is white, mostly from California and Michigan, and their family origins dating back to various parts of europe so time ago. None of that really discribes me though. I've always been a very driven person. The things which I commit a lot of my time and attention to are not always useful or productive. When my mind is on something, I often go for hours thinking or working on it. That is what lead me to math in my own way. If I'm being honest, math is something that given a fixed period of time, I could obsess over. And that obsession grew, every time a new subject was taught, given it wasn't terribly proof-heavy, I would consume it rather rapidly. Once I found calculus, it had seemed that every other math question I had wanted to answer had been opened to me.

During the middle of my undergrad in Mathematics at UCSC, I found myself somewhat without purpose. I had been studying math so long, just because I found it pleasant to study math, that I sort of forgot what I wanted to do with my life. I also realized that knowing how to compute some integrals wasn't what made you the coolest mathematician ever. I wanted something to specialize in, and something to be the best at. I remember looking at courses offered that would satisfy my major requirements and feeling so bored; why was this upper-division course just called ``Algebra''? It was then I realized that Mathematics was a subject so tedious and convuluted that it had lost touch with the real world. 

That same day, I saw the course offering ``Introduction to Fluid Dynamics'' from the Applied Math department, and I became wonderfully curious.  Curious enough such that I realized I wanted to make a change. I took a Dynamical Systems class in Fall 2022 from the Applied Math department as it satisfied a requirement in my current major, and I fell in love with the concept. Proofs and specificity were still part of the subject, but the air of the subject was different. There was a freedom in that course which was never really present in any of the math classes I took. Moreover, the professor for that very course happened to be a part of the Fluids group. Very suddenly, my prior wonder become something tangible, it became conversations. Before that quarter ended, those conversations turned into an application, and I applied to the department's masters program and enrolled in the graduate level dynamical systems course. 

Now, a year later, I'm in the first year of my accelerated masters degree completeing a masters thesis on Stratified Turbulence in Stellar Flows with Pascale Garaud, having already taken 2 graduate level Fluid Dynamics courses at UCSC, and spending a summer in Towson's REU program with Herve Nganguia studying numerical models of propulsion in fluids at micro scales using machine learning. It is sort of amazing how fast your life can change when you find something that really sparks your interest like that. And yet, I feel I'm still missing something. This year will end before I know it and I don't want to end my academic career knowing that there is still more for me to learn in this world. There are simply more classes I want to take, more professors that I want to meet, and potentially more subjects to find and become enamored with. For this reason, I want to continue my education with a PhD in Fluid Dynamics. There are several types of deparments that study this. Some are under Applied/Computational Mathematics programs, while others are found in various Fluid Mechanics or Mechanical Engineering programs. 




\end{document}
