\documentclass{article}
\usepackage{graphicx} % Required for inserting images
\usepackage[margin=1in]{geometry}
\usepackage{amsmath}
\usepackage{verbatim}

\title{DNS of Stratified Turbuluence with Rotation and Stochastic Forcing}
\author{Dante Buhl}
\date{October 2023}

\begin{document}

\maketitle

\section{Previous Work}


\section{Current Work}


\section{Gaussian Processes}
\begin{comment}
    
\end{comment}

My current job is to design a stochastic forcing structure using the Gaussian random process. Gaussian Processes are a way of generating a regression from current data, fitting a line almost if you will. We are using gaussian processes to use the current data to inform a new point going forward in the code. 

The concept of the Gaussian Process is not a novel idea. Its purpose is to generate new points which fit onto an informed window of uncertainty around a given set of initial data. Ultimately, the process samples a gaussian distribution whose mean and covariance matrices are created through the use of precise linear algebra and a kernel chosen to optimize on the desired properties of the gaussian regression. 

The purpose of the Gaussian Process in the context of this work is to create a statistically stationary stochastic forcing in which to perturb and drive eddies in a stable manner as done in (Waite 2004) **SOURCE**. Ultimately blah blah blah. HEhe
\begin{align*}
    G(k, t) &= G_x(k, t) + G_y(k, t) \\
    \vec{G}(k, t) \cdot \vec{k} &= 0
\end{align*}


\section{Code Design and Algorithm Structure}



\end{document}
